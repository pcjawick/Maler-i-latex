% !TEX TS-program = pdflatex
% !TEX encoding = UTF-8 Unicode

% This is a simple template for a LaTeX document using the "article" class.
% See "book", "report", "letter" for other types of document.

\documentclass[11pt]{article} % use larger type; default would be 10pt

\usepackage[utf8]{inputenc} % set input encoding (not needed with XeLaTeX)
\usepackage[pdftex]{graphicx}
\usepackage{sidecap}
\usepackage{wrapfig}

%%% Examples of Article customizations
% These packages are optional, depending whether you want the features they provide.
% See the LaTeX Companion or other references for full information.

%%% PAGE DIMENSIONS
\usepackage{geometry} % to change the page dimensions
\geometry{a4paper} % or letterpaper (US) or a5paper or....
\geometry{margin=0.5in} % for example, change the margins to 2 inches all round
% \geometry{landscape} % set up the page for landscape
%   read geometry.pdf for detailed page layout information

\usepackage{graphicx} % support the \includegraphics command and options

% \usepackage[parfill]{parskip} % Activate to begin paragraphs with an empty line rather than an indent

%%% PACKAGES
\usepackage{booktabs} % for much better looking tables
\usepackage{array} % for better arrays (eg matrices) in maths
\usepackage{paralist} % very flexible & customisable lists (eg. enumerate/itemize, etc.)
\usepackage{verbatim} % adds environment for commenting out blocks of text & for better verbatim
\usepackage{subfig} % make it possible to include more than one captioned figure/table in a single float
\usepackage{titlesec}%Følgende tre lagt til for kryssref og forside
\usepackage{afterpage}
\usepackage{hyperref}
\usepackage{xcolor}
\usepackage{pdfpages}
\usepackage{mdframed}
%some macros


%overskriften
  \definecolor{headingcolor}{RGB}{150,150,150}
  \newcommand{\overskriften}[1]{\par\noindent\colorbox{headingcolor}
  {\parbox{\dimexpr\textwidth-2\fboxsep\relax}{#1}}}
%underoverskriften
  \definecolor{subheadingcolor}{RGB}{150,150,150}
  \newcommand{\underoverskriften}[1]{\par\noindent\colorbox{subheadingcolor}
  {\parbox{\dimexpr\textwidth-2\fboxsep\relax}{#1}}}
%faktaboksen
  \definecolor{faktabokscolor}{RGB}{150,150,150}
  \newcommand{\faktaboksen}[1]{\par\noindent\colorbox{faktabokscolor}
  {\parbox{\dimexpr\textwidth-2\fboxsep\relax}{#1}}}
%smartgoalboksen
  \definecolor{smartgoalscolor}{RGB}{150,150,150}
  \newcommand{\smarteboksen}[1]{\par\noindent\colorbox{smartgoalscolor}
  {\parbox{\dimexpr\textwidth-2\fboxsep\relax}{#1}}}

%%% HEADERS & FOOTERS
\usepackage{fancyhdr} % This should be set AFTER setting up the page geometry
\pagestyle{fancy} % options: empty , plain , fancy
\renewcommand{\headrulewidth}{0pt} % customise the layout...
\lhead{}\chead{}\rhead{}
\lfoot{}\cfoot{\thepage}\rfoot{}

%%% SECTION TITLE APPEARANCE
%\usepackage{sectsty}
%\allsectionsfont{\sffamily\mdseries\upshape} % (See the fntguide.pdf for font help)
% (This matches ConTeXt defaults)

%%% ToC (table of contents) APPEARANCE
\usepackage[nottoc,notlof,notlot]{tocbibind} % Put the bibliography in the ToC
\usepackage[titles,subfigure]{tocloft} % Alter the style of the Table of Contents
\renewcommand{\cftsecfont}{\rmfamily\mdseries\upshape}
\renewcommand{\cftsecpagefont}{\rmfamily\mdseries\upshape} % No bold!


\mdfdefinestyle{mystyle}{
    backgroundcolor=black!20
}


%%% END Article customizations

%%% The "real" document content comes below...

\title{Nedre Eiker kommune - Øvre Eiker kommune \\ Rapport om SAMRUS prosjektet}
\author{Pål Ager-Wick}
%\date{} % Activate to display a given date or no date (if empty),
         % otherwise the current date is printed 


\begin{document}




%%%\setcounter{section}{0}


              \renewcommand{\abstractname}{\overskriften{\textsf{\Huge{Prosjektbeskrivelse}}}}
              %\overskriften{\textsc{Extra Curricular Achievements}}



             \begin{abstract}%%Lage bredde!!! og skriftboks på høyresiden
           	  	
           	  	
           	  	\begin{wrapfigure}{left}{0.3\textwidth}
  \begin{center}
   %\includegraphics[width=0.48\textwidth]{gull}
   I will include a lot of text here. Just tons of it jahdkj sdlfjh åpe .mndf .n,mdf sdfi n sdfn  lsdkf ø jsdfk lsdf lk lsdf lsf spf p fsp sp sp fsp sfp spf psf psf ps fspf ps fps fps fp sp fps fps fps fps .
  \end{center}
  						%\caption{A gull}
			\end{wrapfigure}
			
			%\begin{SCfigure}
  			%	\centering
  				
 				 %\caption{ ... caption text ... }
				  %\includegraphics[width=0.3\textwidth]%
    				%{Giraff_picture}% picture filename
			%\end{SCfigure}
         ipsum ipsum suadnadkfjnsjkdnf akjkd f dfjk adhauwe0iw w ejc dpdøøf nfnapasøldk asdøI will include a lot of text here. Just tons of it jahdkj sdlfjh åpe .mndf .n,mdf sdfi n sdfn  lsdkf ø jsdfk lsdf lk lsdf lsf spf p fsp sp sp fsp sfp spf psf psf ps fspf ps fps fps fp sp fps fps fps fps .I will include a lot of text here. Just tons of it jahdkj sdlfjh åpe .mndf .n,mdf sdfi n sdfn  lsdkf ø jsdfk lsdf lk lsdf lsf spf p fsp sp sp fsp sfp spf psf psf ps fspf ps fps fps fp sp fps fps fps fps .I will include a lot of text here. Just tons of it jahdkj sdlfjh åpe .mndf .n,mdf sdfi n sdfn  lsdkf ø jsdfk lsdf lk lsdf lsf spf p fsp sp sp fsp sfp spf psf psf ps fspf ps fps fps fp sp fps fps fps fps .I will include a lot of text here. Just tons of it jahdkj sdlfjh åpe .mndf .n,mdf sdfi n sdfn  lsdkf ø jsdfk lsdf lk lsdf lsf spf p fsp sp sp fsp sfp spf psf psf ps fspf ps fps fps fp sp fps fps fps fps .I will include a lot of text here. Just tons of it jahdkj sdlfjh åpe .mndf .n,mdf sdfi n sdfn  lsdkf ø jsdfk lsdf lk lsdf lsf spf p fsp sp sp fsp sfp spf psf psf ps fspf ps fps fps fp sp fps fps fps fps .

              \end{abstract}

              \renewcommand\partname{\overskriften{\textsf{\Huge{Del}}}}
             % \renewcommand{\chaptername}{Del}
              \renewcommand\contentsname{\overskriften{\textsf{\Huge{Innhold}}}}
              \renewcommand\listfigurename{\overskriften{\textsf{\Huge{Illustrasjoner}}}}
              \renewcommand\tablename{\overskriften{\textsf{\Huge{Tabell}}}}
		\renewcommand\listtablename{Tabeller}
              \renewcommand\figurename{\overskriften{\textsf{\Huge{Illustrasjon}}}}
            %  \renewcommand{\bibname}{Kilder:}

             % \tableofcontents

            %  \begin{figure}
              		1-2-3
            %  \end{figure}


                % Her begynner kildehenvisningene
              
         %     \begin{thebibliography}{99}

              




          %   \end{thebibliography}

         %    \listoffigures            
             

              %  \appendix
              

              \end{document}
